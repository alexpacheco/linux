%% Tikz File System Heirarchy
 \tikzset{
    invisible/.style={opacity=0},
    visible on/.style={alt={#1{}{invisible}}},
    alt/.code args={<#1>#2#3}{%
      \alt<#1>{\pgfkeysalso{#2}}{\pgfkeysalso{#3}} % \pgfkeysalso doesn't change the path
    },
  }

\begin{frame}{File System Hierarchy}
\tikzset{every node/.append style={scale=0.4}}

\begin{tikzpicture}[scale=0.4]
  \path[mindmap,concept color=lubrown,text=lucream,
  level 1 concept/.append style={every child/.style={concept color=luorange,text=black},sibling angle=-15},
  level 2 concept/.append style={every child/.style={concept color=luorange2,text=black}, sibling angle=-40},
  level 3 concept/.append style={every child/.style={concept color=lulime,text=black}}]
    node[concept] {root directory (/)}
    [clockwise from=180]
    child[style={concept}, level distance=8.5cm, visible on=<2->] { node[concept] {bin}}
    child[concept, visible on=<3->] { node[concept] {boot}}
    child[style={concept}, level distance=8.5cm, visible on=<4->] { node[concept] {dev}}
    child[concept, visible on=<5->] { node[concept] {etc}}
    child[style={concept}, level distance=8.5cm, visible on=<6->] { node[concept] {home}
		[counterclockwise from=-90]
      	child { node[concept] {alp514} 
        	[clockwise from=0]
            child { node[concept] {.bashrc}}
            child { node[concept] {bin}}
            child { node[concept] {Documents}}
            child { node[concept] {packages}}
            child { node[concept] {$\cdots$}}
        }
      	child[visible on=<16->] { node[concept] {sma310} }
      	child[visible on=<16->] { node[concept] {$\cdots$} }
	}
    child[concept, visible on=<7->] { node[concept] {lib64 \& lib}}
    child[style={concept}, level distance=8.5cm, visible on=<8->] { node[concept] {mnt}}
    child[concept, visible on=<9->] { node[concept] {proc}}
    child[style={concept}, level distance=8.5cm, visible on=<10->] { node[concept] {sbin}}
    child[concept, visible on=<11->] { node[concept] {tmp}}
    child[style={concept}, level distance=11.5cm, visible on=<12->] { node[concept] {usr}
		[counterclockwise from=15]
      	child { node[concept] {bin} }
      	child { node[concept] {lib64 \& lib} }
      	child { node[concept] {local} }
      	child { node[concept] {include} }
      	child { node[concept] {sbin} }
      	child { node[concept] {share} }
        }
    child[concept, visible on=<13->] { node[concept] {var}}
    child[style={concept}, level distance=8.5cm, visible on=<14->] { node[concept] {zhome}
		[counterclockwise from=0]
      	child { node[concept] {Apps} 
        	[clockwise from=50]
            child { node[concept] {matlab}}
            child { node[concept] {gcc}}
            child { node[concept] {intel}}
            child { node[concept] {pgi}}
        }
      	child { node[concept] {alp514} }
      	child { node[concept] {sma310} }
      	child { node[concept] {$\cdots$} }
	}
;
\end{tikzpicture}

\begin{itemize}
\only<1>{\item All files are arranged in a hierarchial structure, like an inverted tree.}
\only<1>{\item The top of the hierarchy is traditionally called \textbf{root} (written as a slash / )}
\only<2>{\item contains files that are essential for system operation, available for use by all users.}
\only<3>{\item contains bootable kernel and bootloader}
\only<4>{\item contains various devices such as hard disk, CD-ROM drive etc}
\only<5>{\item contains various system configurations}
\only<6>{\item contains home directories of all users. This is the directory where you are at when you login to a Linux/UNIX system.}
\only<7>{\item contains libraries that are essential for system operation, available for use by all users.}
\only<8>{\item directories where disk drives are mounted}
\only<9>{\item process information pseudo-file system containing runtime system information (e.g. system memory, devices mounted, hardware configuration, etc).}
\only<9>{\item can be regarded as a control and information centre for the kernel.}
\only<10>{\item same as bin but only accessible by \textbf{root}}
\only<11>{\item temporary file storage}
\only<12>{\item contains user documentations, binaries, libraries etc}
\only<13>{\item used to store files which change frequently (system level not user level)}
\only<14>{\item where we install applications common to all HPC systems}
\only<15>{\item Installing your own OS: /bin,/lib\{64\},/etc,/dev and /sbin must be on the same partition.}
\only<16>{\item UNIX like OS's are designed for multi user environments i.e. multiple users can exist on the system.}
\only<16>{\item Special user called \textbf{root} is the administrator and has access to all files in the system.}
\end{itemize}
\end{frame}

