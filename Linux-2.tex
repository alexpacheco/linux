\documentclass[10pt,t]{beamer}

\input{mypreamble}
\beamertemplateballitem
\usepackage{tabu}

\newcolumntype{a}{>{\columncolor{lulime}}c}
\newcolumntype{b}{>{\columncolor{lulime!50}}c}
\newcolumntype{d}{>{\columncolor{lulime!40}}c}
\newcolumntype{e}{>{\columncolor{lulime}}l}
\newcolumntype{f}{>{\columncolor{lulime!50}}l}

%% Tikz Distro Watch Table
% Defining some symbols:
\newcommand*\head[1]{\textbf{#1}}
% The table environment:
\newenvironment{matrixtable}[4]{%
  \begin{tikzpicture}[matrix of nodes/.style={
    execute at begin cell=\node\bgroup\strut,
    execute at end cell=\egroup;}]
  \matrix (m) [matrix of nodes,top color=blue!20,
    bottom color=blue!80,draw=white,
    nodes={draw,top color=blue!10,bottom color=blue!35,
    draw,inner sep=2pt,minimum height=3.1ex},
    column sep=1ex,row sep=0.6ex,inner sep=2ex,
    rounded corners,column 1/.style={minimum width=#1},
    column 2/.style={minimum width=#2},
    column 3/.style={minimum width=#3},
    column 4/.style={minimum width=#4}]}%
{;\end{tikzpicture}}

\title{Introduction to Linux}
\subtitle{Basic Commands \& Environment}
\author{Alexander B. Pacheco}
\institute{\href{http://researchcomputing.lehigh.edu}{Research Computing}}%\\[2pt] \href{http://www.lehigh.edu}{Lehigh University}}
%\date{September 15, 2015}
\date{ }

% Delete this, if you do not want the table of contents to pop up at
% the beginning of each subsection:
\AtBeginSection[]
{
  \begingroup
  \setbeamertemplate{background canvas}[vertical shading][bottom=lubrown,top=lubrown]
  \setbeamertemplate{footline}[myfootline] 
  \setbeamertemplate{section page}[mysection]
  \frame[c]{
    \sectionpage
  }
  \endgroup
}

\titlegraphic{\includegraphics[scale=0.5]{lu}}
\begin{document}

\begin{frame}[c]
  \titlepage
\end{frame}

\footnotesize
\begin{frame}{Outline}
  \tableofcontents
\end{frame}

\section{Introduction}
\begin{frame}
  \frametitle{What is Linux?}
  \begin{itemize}
    \item Linux is an operating system that evolved from a kernel created by Linus Torvalds when he was a student at the University of Helsinki. 
    \item It's meant to be used as an alternative to other operating systems, Windows, Mac OS, MS-DOS, Solaris and others. 
    \item Linux is the most popular OS used in a Supercomputer\let\thefootnote\relax\footnote{\tiny \url{http://www.top500.org/statistics/list/}}\let\thefootnote\relax\footnote{\tiny June 2015 List}
      \begin{center}
  \begin{tikzpicture}
    \node (tbl) {
      \begin{tabularx}{0.42\textwidth}{ccc}
        \arrayrulecolor{black}
        \textcolor{white}{\textbf{OS Family} }& \textcolor{white}{\textbf{Count}} &\textcolor{white}{\textbf {Share \%}}\\
        Linux \rule{0pt}{3.5ex} & 485 & 97 \\
        Unix & 13 & 2.6 \\
        Mixed & 1 & 0.2 \\
        Windows & 1 & 0.2 \\
        [1.0ex]
    \end{tabularx}};
    \begin{pgfonlayer}{background}
      %\draw[rounded corners,top color=blue!30!black,bottom color=blue!10!white,
      \draw[rounded corners,top color=lupurple,bottom color=lupurple,
        draw=lubrown!30] ($(tbl.north west)+(0.14,0)$)
      rectangle ($(tbl.north east)-(0.13,0.9)$);
      %\draw[rounded corners,top color=green!5,bottom color=green!5,draw=green!5]
      \draw[rounded corners,top color=lulime,bottom color=lulime,draw=lulime]
      ($(tbl.north east)-(0.13,0.6)$)
      rectangle ($(tbl.south west)+(0.13,0.2)$);
    \end{pgfonlayer}
  \end{tikzpicture}
\end{center}

    \item All Top 500 Supercomputers were Linux based in the November 2017 List.
    \item If you are using a Supercomputer/High Performance Computer for your research, it will be based on a *nix OS.
    \item It is required/neccessary/mandatory to learn Linux Programming (commands, shell scripting) if your research involves use of High Performance Computing or Supercomputing resources.
  \end{itemize}
\end{frame}

\begin{frame}
  \frametitle{Where is Linux used?}
  \begin{itemize}
    \item Linux distributions are tailored to different requirements such as
    \begin{enumerate}
      {%\scriptsize
        \item Server
        \item Desktop
        \item Workstation
        \item Routers
        \item Embedded devices
        \item Mobile devices (Android is a Linux-based OS)
      }
    \end{enumerate}
    \item Almost any software that you use on windows has a roughly equivalent software on Linux, most often multiple equivalent software
    \item[e.g.] Microsoft Office equivalents are OpenOffice.org, LibreOffice, KOffice
    \item For complete list, visit \url{http://wiki.linuxquestions.org/wiki/Linux_software_equivalent_to_Windows_software}
    \item Linux offers you freedom, to choose your desktop environment, software.
  \end{itemize}
\end{frame}

\begin{frame}
  \frametitle{What is a Linux OS, Distro, Desktop Environment?}
  \begin{itemize}
    \item Many software vendors release their own packaged Linux OS (kernel, applications) known as distribution
    \item Linux distribution = Linux kernel + GNU system utilities and libraries + Installation scripts + Management utilities etc.
    \begin{enumerate}
      {%\scriptsize
        \item Debian, Ubuntu, Mint
        \item Red Hat, Fedora, \textbf{CentOS}
        \item Slackware, openSUSE, SLES, SLED
        \item Gentoo
      }
    \end{enumerate}
    \item Application packages on Linux can be installed from source or from customized packages
    \begin{enumerate}
      {%\scriptsize
        \item deb: Debian based distros e.g. Debian, Ubuntu, Mint
        \item rpm: Red Hat based distros, Slackware based distros.
      }
    \end{enumerate}
    \item Linux distributions offer a variety of desktop environment.
    \begin{enumerate}
      {%\scriptsize
        \item K Desktop Environment (KDE)
        \item GNOME 
        \item Xfce
        \item Lightweight X11 Desktop Environment (LXDE)
        \item Cinnamon
        \item MATE
        \item Dynamic Window Manager
      }
    \end{enumerate}
  \end{itemize}
\end{frame}

\begin{frame}[fragile]
  \frametitle{Difference between Shell and Command}
  \begin{block}{What is a Shell?}
    \begin{itemize}
    \item The command line interface is the primary interface to Linux/Unix operating systems.
    \item Shells are how command-line interfaces are implemented in Linux/Unix.
    \item Each shell has varying capabilities and features and the user should choose the shell that best suits their needs.
    \item The shell is simply an application running on top of the kernel and provides a powerful interface to the system.
    \end{itemize}
  \end{block}
  \begin{exampleblock}{What is a command and how do you use it?}
    \begin{itemize}
    \item \textbf{command} is a directive to a computer program acting as an interpreter of some kind, in order to perform a specific task.
    \item \textbf{command prompt} (or just \textbf{prompt}) is a sequence of (one or more) characters used in a command-line interface to indicate readiness to accept commands.
    \item Its intent is to literally prompt the user to take action. 
    \item A prompt usually ends with one of the characters \$, \%, \#, :, $>$ and often includes other information, such as the path of the current working directory.
      %      \item[$\mybigstar$] Virtual Image: \Verb[formatcom=\color{lubrown},fontseries=b,commandchars=\\\{\}]|[user@localhost ~]\$|
      %      \item[$\mybigstar$] QueenBee: \texttt{[apacheco@qb4 $\sim$]\$}
      %      \item[$\mybigstar$] Mac OSX in \Verblubrown{tcsh}: \Verb[formatcom=\color{lubrown},fontseries=b,commandchars=\\\{\}]|[c8-bc-c8-ee-b8-9e:~] apacheco\%|
      %      \item Each \textbf{command} consists of three parts: name, options, arguments
      %      \item[] \Verb[formatcom=\color{lubrown},fontseries=b,commandchars=\\\{\}]|[user@localhost ~]\$ command options arguments|
    \end{itemize}
  \end{exampleblock}
\end{frame}
  
\begin{frame}
  \frametitle{Types of Shell}
    \begin{itemize}
      \item[\texttt{sh}]: Bourne Shell
      \begin{enumerate}
        {%\scriptsize
          \item[$\vardiamond$] Developed by Stephen Bourne at AT\&T Bell Labs
        }
      \end{enumerate}
      \item[\texttt{csh}]: C Shell
      \begin{enumerate}
        {%\scriptsize
          \item[$\vardiamond$] Developed by Bill Joy at University of California, Berkeley
        }
      \end{enumerate}
      \item[\texttt{ksh}]: Korn Shell
      \begin{enumerate}
        {%\scriptsize
          \item[$\vardiamond$] Developed by David Korn at AT\&T Bell Labs
          \item[$\vardiamond$] backward-compatible with the Bourne shell and includes many features of the C shell
        }
      \end{enumerate}
      \item[\texttt{bash}]: Bourne Again Shell
      \begin{enumerate}
        {%\scriptsize
          \item[$\vardiamond$] Developed by Brian Fox for the GNU Project as a free software replacement for the Bourne shell (sh).
          \item[$\vardiamond$] Default Shell on Linux and Mac OSX
          \item[$\vardiamond$] The name is also descriptive of what it did, bashing together the features of sh, csh and ksh
        }
      \end{enumerate}
      \item[\texttt{tcsh}]: TENEX C Shell
      \begin{enumerate}
        {%\scriptsize
          \item[$\vardiamond$] Developed by Ken Greer at Carnegie Mellon University 
          \item[$\vardiamond$] It is essentially the C shell with programmable command line completion, command-line editing, and a few other features.
        }
      \end{enumerate}
    \end{itemize}
\end{frame}

\begin{frame}
  \frametitle{Shell Comparison}
  \begin{columns}
    \column{0.7\textwidth}
    \begin{center}
      \begin{tikzpicture}
        \node (tbl) {
          \begin{tabularx}{\textwidth}{cccccc}
            \arrayrulecolor{black}
            \textcolor{white}{\textbf{Software} }& \textcolor{white}{\textbf{sh}} &\textcolor{white}{\textbf {csh}} & \textcolor{white}{\textbf{ksh}} & \textcolor{white}{\textbf{bash}} & \textcolor{white}{\textbf{tcsh}} \\
            Programming Language\rule{0pt}{3.5ex} & \cmark & \cmark & \cmark & \cmark & \cmark \\
            Shell Variables & \cmark & \cmark & \cmark & \cmark & \cmark \\
            Command alias & \xmark & \cmark & \cmark & \cmark & \cmark \\
            Command history & \xmark & \cmark & \cmark & \cmark & \cmark \\
            Filename completion & \xmark & \smark & \smark & \cmark & \cmark \\
            Command line editing & \xmark & \xmark & \smark & \cmark & \cmark \\
            Job control & \xmark & \cmark & \cmark & \cmark & \cmark \\
            [1.0ex]
        \end{tabularx}};
        \begin{pgfonlayer}{background}
          %\draw[rounded corners,top color=blue!30!black,bottom color=blue!10!white,
          \draw[rounded corners,top color=lupurple,bottom color=lupurple,
            draw=lubrown!30] ($(tbl.north west)+(0.14,0)$)
          rectangle ($(tbl.north east)-(0.13,0.9)$);
          %\draw[rounded corners,top color=black,bottom color=lubrown!20,
          %   middle color=lubrown!20,draw=lubrown!20] ($(tbl.south west)
          %   +(0.12,0.5)$) rectangle ($(tbl.south east)-(0.12,0)$);
          %\draw[rounded corners,top color=green!5,bottom color=green!5,draw=green!5]
          \draw[rounded corners,top color=lulime,bottom color=lulime,draw=lulime]
          ($(tbl.north east)-(0.13,0.6)$)
          rectangle ($(tbl.south west)+(0.13,0.2)$);
        \end{pgfonlayer}
      \end{tikzpicture}
      \begin{itemize}
        \item[\cmark]: Yes
        \item[\xmark]: No
        \item[\smark]: Yes, not set by default
        \item[] {\fontsize{6}{7}\selectfont\url{http://www.cis.rit.edu/class/simg211/unixintro/Shell.html}}
      \end{itemize}
    \end{center}
  \end{columns}
\end{frame}

\section{Linux File System Heirarchy}
\begin{frame}
  \frametitle{Directory Structure}
%    \vspace{-0.5cm}
  \begin{itemize}
    \item All files are arranged in a hierarchial structure, like an inverted tree.
    \item The top of the hierarchy is traditionally called \textbf{root} (written as a slash / )
  \end{itemize}
\tikzstyle{every node}=[draw=black,thick,anchor=west]
\tikzstyle{selected}=[draw=red,fill=red!30]
\tikzstyle{optional}=[dashed,fill=gray!50]
\begin{tikzpicture}[%
  linux/.style={rectangle,draw,fill=blue!20},
  hpc/.style={rectangle,draw,fill=red!80,rounded corners=.8ex},
  grandchild/.style={grow=down,xshift=1em,anchor=west,
    edge from parent path={(\tikzparentnode.south) |- (\tikzchildnode.west)}},
  first/.style={level distance=12ex,sibling distance=10em,xshift=-6em},
  second/.style={level distance=8ex,sibling distance=6em,xshift=-1.5em},
  third/.style={level distance=26ex,sibling distance=7.5em,xshift=-2em},
  fourth/.style={level distance=8ex},
  fifth/.style={level distance=14ex},
  sixth/.style={level distance=20ex},
  seventh/.style={level distance=26ex},
  eighth/.style={level distance=32ex},
  level 1/.style={sibling distance=5.1em},
  scale=0.58]
    % Parents
    \coordinate
    child[grow=down,level distance=0ex]{node[linux]{ / }}
    [edge from parent fork down]
    % Children and grandchildren
    child {node[linux] {bin}}
    child {node[linux] {boot}}
    child {node[linux] {dev}}
    child {node[linux] {etc}}
    child {node[linux] {home}
      child [first] {node[linux] {user1}
          child[grandchild,fourth] {node[linux] {Desktop}}
          child[grandchild,fifth] {node[linux] {Documents}}
          child[grandchild,sixth] {node[linux] {Downloads}}
          child[grandchild,seventh] {node[linux] {Public}}
          child[grandchild,eighth] {node[linux] {$\cdots$}}
       }
      child [first] {node[linux] {user2}
       }
      child [first] {node[linux] {$\cdots$}}
    }
    child [missing] {}
    child {node[linux] {lib}}
    child {node[linux] {mnt}}
    child {node[linux] {tmp}}
    child {node[linux] {usr}
      child [third] {node[linux] {bin}}
      child [third] {node[linux] {lib}}
      child [third] {node[linux] {local}}
      child [third] {node[linux] {include}}
      child [third] {node[linux] {share}}
    }
    child {node[linux] {var}}
    child {node[hpc] {share}
      child [second] {node[hpc] {Apps}}
    }
%      child [second] {node[hpc] {user1}}
%      child [second] {node[hpc] {user2}}
%      child [second] {node[hpc] {$\cdots$}}
    ;
\end{tikzpicture}

\begin{itemize}
  \item Want to use Sol software on your local resource 
  \item[] Visit \url{https://go.lehigh.edu/linux}
  \item For list of software installed on Sol
  \item[] Visit \url{https://go.lehigh.edu/solsoftware}
\end{itemize}
\end{frame}

\begin{frame}
  \frametitle{Relative \& Absolute Path}
  \begin{itemize}
    \item \textit{Path} means a position in the directory tree.
    \item You can use either the \textit{relative path} or \textit{absolute path}
    \item In \textit{relative path} expression
    \begin{itemize}
      {%\scriptsize
        \item . (one dot or period) is the current working directory
        \item .. (two dots or periods) is one directory up
        \item You can combine . and .. to navigate the file system hierarchy.
        \item the path is not defined uniquely and does depend on the current path.
        \item \texttt{../../tmp} is unique only if your current working directory is your home directory.
      }
    \end{itemize}
    \item In \textit{absolute path} expression 
    \begin{itemize}
      {%\scriptsize
        \item the path is defined uniquely and does not depend on the current path
        \item \texttt{/tmp} is unique since /tmp is the \textit{abolute path}
      }
    \end{itemize}
  \end{itemize}
\end{frame}


\section{Basic Commands}

\begin{frame}[fragile]{How to get more information with Linux}
  \begin{enumerate}
  \item \Verblubrown{man} shows the manual for a command or program.
    \begin{itemize}
    \item The manual is a file that shows you how to use the command and list the different options for the command in question.
    \item Usage: \Verblubrown{man [command]}
    \item Example: \Verblubrown{man ls{\Enter}}
    \end{itemize}
  \item \Verblubrown{info} similar to man, with a more robust structure for linking pages together.
    \begin{itemize}
    \item Usage: \Verblubrown{info [command]}
    \item Example: \Verblubrown{info man{\Enter}}
    \end{itemize}
  \item[] To quit from \Verblubrown{man} or \Verblubrown{info}, type \Verblubrown{q}
  \item \Verblubrown{apropos} shows you all of the man pages that may shed some light on a certain command.
    \begin{itemize}
    \item Usage: \Verblubrown{appropos [keyword]}
    \item Example: \Verblubrown{appropos editor{\Enter}}
    \end{itemize}
  \item \Verblubrown{echo} is used to print information to screen
    \begin{itemize}
    \item Usage: \Verblubrown{echo [arguments]}
    \item Example: \Verblubrown{echo Welcome to LTS Seminar on Linux Commands}
    \end{itemize}
  \end{enumerate}
% leave space for info command
\end{frame}



%\begin{frame}[fragile,allowframebreaks]
%  \frametitle{Input \& Output Commands}
%  \begin{itemize}
%    \item The basis I/O statements are \Verblubrown{echo} for displaying output to screen and \Verblubrown{read} for reading input from screen/keyboard/prompt
%    \item The \Verblubrown{read} statement takes all characters typed until the \Verblubrown{{\Enter}} key is pressed and stores them into a variable.
%    \item Usage: \Verblubrown{read <variable name>}
%    \item Example: \Verblubrown{read name{\Enter}}
%    \item[] \Verbblue[it]{Alex Pacheco{\Enter}}
%    \item In the above example, the name that you enter in stored in the variable \Verblubrown{name}.
%    \item[]
%    \item The \Verblubrown{echo arguments} command will print \Verblubrown{arguments} to screen or standard output. 
%    \item \Verblubrown{arguments} can be a (single or multiple) variable, string of characters or numbers.
%      \framebreak
%    \item Examples:
%      \begin{enumerate}
%        {%\footnotesize
%          \item \Verblubrown{echo \$LD\_LIBRARY\_PATH \$LD\_INCLUDE\_PATH{\Enter}}
%          \item \Verblubrown{echo Welcome to HPC {\quad} Training{\Enter}}
%        }
%      \end{enumerate}
%    \item By default, \Verblubrown{echo} eliminates redundant whitespace (multiple spaces and tabs) and replaces it with a single whitespace between arguments. 
%    \item To include redundant whitespace, enclose the arguments within double quotes
%    \item[e.g.] \Verblubrown{echo "Welcome to HPC{\quad\quad}Training"{\Enter}}
%  \end{itemize}
%
%  \begin{exampleblock}{Exercise}
%    \begin{itemize}
%     \item Print out the variable you created a few slides back
%      \item[] \Verblubrown{echo \$MYNAME{\Enter}}
%      \item[] \Verblubrown{echo \$Greet{\Enter}}
%      \item Read a variable for greeting message
%      \item[] \Verblubrown{read message{\Enter}}
%      \item[] \Verbblue[it]{Welcome to HPC{\Enter}}
%      \item Combine and print your name, the greeting and the message
%      \item[] \Verblubrown{echo \$Greet \$MYNAME \$message{\Enter}}
%      \item What is the output of the following command?
%      \item[] \Verblubrown{echo \$Greet \$MYNAME, \$message Training{\Enter}}
%    \end{itemize}
%  \end{exampleblock}
%\end{frame}

\begin{frame}[c,fragile,allowframebreaks]{Basic Commands}
  \begin{enumerate}
  \item \Verblubrown{pwd}
    \begin{itemize}
    \item prints the current working directory.
    \item Usage: \Verblubrown{pwd}
    \item Example: \Verblubrown{pwd{\Enter}}
    \end{itemize}
  \item \Verblubrown{cd}
    \begin{itemize}
    \item allows one to change directory
    \item argument is the path (relative or absolute) of the directory you want to change to
    \item Usage: \Verblubrown{cd [destination]}
    \item Example: \Verblubrown{cd /tmp{\Enter}}
    \item The default destination directory is your home directory.
    \item i.e. If you type \Verblubrown{cd{\Enter}}, you will end up in your home directory.
    \item If you want to go back to the previous directory, type \Verblubrown{cd - {\Enter}}
    \end{itemize}
    \framebreak
  \item \Verblubrown{ls}
    \begin{itemize}
    \item lists the contents of a directory.
    \item Usage: \Verblubrown{ls <options> <path>}
    \item Example: \Verblubrown{ls{\Enter}}
    \item The current working directory is the default path.
    \item To list contents of another directory specify the path, relative or absolute
    \item Common options to the \Verblubrown{ls} command 
    \item[] \Verblubrown{-l}: show long listing format
    \item[] \Verblubrown{-a}: show hidden files
    \item[] \Verblubrown{-r}: reverse order while sorting
    \item[] \Verblubrown{-t}: show modification times
    \item[] \Verblubrown{-h}: use file sizes in SI units (bytes, kilobytes, megabytes etc ) default is bytes
    \end{itemize}
    \framebreak
  \item \Verblubrown{alias}
    \begin{itemize}
    \item  create a shortcut to another command or name to execute a long string.
    \item Usage 
    \item[] \Verblubrown{bash/sh/ksh}: \Verblubrown{alias <name>="<actual command>"}
    \item[] \Verblubrown{csh/tcsh}: \Verblubrown{alias <name> "<actual command>"}
    \item Example: 
    \item[] \Verblubrown{bash/sh/ksh}: \Verblubrown{alias lla="ls -al"}
    \item[] \Verblubrown{csh/tcsh}: \Verblubrown{alias lls "ls -al"}
    \item The \Verblubrown{alias} command is very useful tool to create shortcuts to other commands and is most often used by paranoid users to prevent accidental deletion of files. 
    \item \Verblubrown{unalias} is a command to remove an alias.
    \item Usage: \Verblubrown{unalias <name>}
    \item Example: \Verblubrown{unalias lla} will remove the shortcut to \Verblubrown{ls -al}
    \end{itemize}
    \framebreak
  \item \Verblubrown{mkdir} 
    \begin{itemize}
    \item create a directory
    \item Usage: \Verblubrown{mkdir <options> <directoryname>}
    \item Example: \Verblubrown{mkdir -p \$HOME/test/testagain{\Enter}}
    \item By default, the directory is created in the current directory or in a path relative to the current directory
    \item The \Verblubrown{-p} option will create intermediate directories if they do not exist.
    \item[e.g.] If the directory \Verblubrown{test} does not exist in \Verblubrown{\$HOME}, then 
    \item[] \Verblubrown{mkdir \$HOME/test/testagain} will fail. 
    \item[] The \Verblubrown{-p} option will create the \Verblubrown{test} directory within \Verblubrown{\$HOME} and then create \Verblubrown{testagain} within the newly created \Verblubrown{test} directory
    \end{itemize}
    \framebreak
  \item \Verblubrown{cp}
    \begin{itemize}
    \item copy a file or directory
    \item Usage: \Verblubrown{cp <options> <source(s)> <destination>}
    \item Example: \Verblubrown{cp \$HOME/.bashrc ../../tmp{\Enter}}
    \item Common options to \Verblubrown{cp} command:
    \item[] \Verblubrown{-r}: copy recursively, required when copying directories.
    \item[] \Verblubrown{-i}: prompt if file exists on destination and can be copied over.
    \item[] \Verblubrown{-p}: preserve file access times, ownership etc.
    \item If there are more than one source files, then the destination (i.e. last entry or file) must be a directory.
    \item If the source(s) is(are) a file(s) and the destination is a directory, then the file(s) will be copied into the directory
    \item[e.g.]\Verblubrown{cp file1 file2 dir1{\Enter}}
    \item[] \Verblubrown{dir1} will contain the files \Verblubrown{file1} and \Verblubrown{file2}
    \item[] If \Verblubrown{dir1} is a file, then the above command will fail
    \end{itemize}
    \framebreak
  \item \Verblubrown{rm} 
    \begin{itemize}
    \item removes or deletes a file or directory
    \item Usage: \Verblubrown{rm <options> <file or directory>}
    \item Example: \Verblubrown{rm \$HOME/tmpfile{\Enter}}
    \item Common options to \Verblubrown{rm} command:
    \item[] \Verblubrown{-r}: remove recursively, required when copying directories.
    \item[] \Verblubrown{-i}: prompt if file really needs to be deleted
    \item[] \Verblubrown{-f}: force remove overrides the \Verblubrown{-i} option
    \item {\color{red}{\large\textsc{be careful while using the }\textbf{rm}\textsc{ command, deleted files cannot be recovered}}}
    \item To be on the safe side, create an \Verblubrown{alias} to the \Verblubrown{rm} command and only use the \Verblubrown{-f} option only if you are sure you want to delete the file or directory
    \item[] \Verblubrown{sh/ksh/bash}: \Verblubrown{alias rm="rm -i"}
    \item[] \Verblubrown{csh/tcsh}   : \Verblubrown{alias rm 'rm -i'}
    \item delete empty directories using the \Verblubrown{rmdir} command.
    \end{itemize}
    \framebreak
  \item \Verblubrown{mv} 
    \begin{itemize}
      \item moves or renames a file or directory
      \item Usage: \Verblubrown{mv <options> <source> <destination>}
      \item Example: \Verblubrown{mv test test1}
      \item If there are more than one source file, then the last file is the destination and must be a directory.
      \item Use the \Verblubrown{-i} option to prompt if a file or directory will be overwritten.
      \item If the source(s) is(are) a file(s) and the destination is a directory, then the file(s) will be copied into the directory.
      \item[e.g.]\Verblubrown{mv file1 file2 dir1{\Enter}}
      \item[] \Verblubrown{dir1} will contain the files \Verblubrown{file1} and \Verblubrown{file2}
      \item[] If \Verblubrown{dir1} is a file, then the above command will fail
    \end{itemize}
  \end{enumerate}
\end{frame}

\begin{frame}[fragile]{Pager Commands}
  \begin{itemize}
    \item To display a file to screen, *nix provides three commands at your disposal
  \end{itemize}
  \begin{enumerate}
    \item \Verblubrown{cat}: Show contents of a file.
    \item \Verblubrown{more}: Display contents one page at a time.
    \item \Verblubrown{less}: Display contents one page at a time but allow forward/backward scrolling
  \end{enumerate}
  \begin{itemize}
  \item Usage: \Verblubrown{cat/more/less <options> <filename>}
  \item Example: \Verblubrown{cat .bashrc}
  \item To scroll forward in \Verblubrown{more} or \Verblubrown{less}, use the space bar, \texttt{CNTRL-f/d} or "Page Down" key.
  \item To scroll backwards in \Verblubrown{less} use \texttt{CNTRL-b/u} or "Page Up".
  \item To quit from \Verblubrown{more} or \Verblubrown{less}, use \texttt{q}
  \item A rarely used command, \Verblubrown{tac} does the opposite of \Verblubrown{cat} i.e. show contents of a file in reverse.
  \end{itemize}
\end{frame}

\begin{frame}[fragile,allowframebreaks]{Other Commands}
  \begin{description}
    \item[passwd:] change password %(\alert{does not work on LSU HPC and LONI systems})
    \item[chsh:] change default shell %(\alert{does not work on LSU HPC and LONI systems})
    \item[df:] report disk space usage by filesystem
    \item[du:] estimate file space usage - space used under a particular directory or files on a file system.
    \item[sudo:] run command as root (\alert{only if you have access})
    \item[mount:] mount file system (\alert{root only})
    \item[umount:] unmount file system (\alert{root only})
    \item[shutdown:] reboot or turn off machine (\alert{root only})
    \item[top:] Produces an ordered list of running processes
    \item[htop:] An interactive process viewer for Linux (\alert(not installed by default))
    \item[free:] Display amount of free and used memory in the system
    \item[file:] Determine file type
    \item[touch:] change file timestamps or create file if not present
    \item[date:] display or set date and time
    \item[find]: Find a file \Verblubrown{find /dir/to/search -name file-to-search}
    \item[wc:] Count words, lines and characters in a file \Verblubrown{wc -l .bashrc}
    \item[grep:] Find patterns in a file \Verblubrown{grep alias .bashrc}
    \item[awk:] File processing and report generating \Verblubrown{awk '\{print \$1\}' file1}
    \item[sed:] Stream Editor \Verblubrown{sed 's/home/HOME/g' .bashrc}
    \item[set:] manipulate environment variables \Verblubrown{set -o emacs}
    \item[ln:] Link a file to another file \Verblubrown{ln -s file1 file2}
    \item[head:] Display first lines of a file \Verblubrown{head file1}
    \item[tail:] Display last lines of a file \Verblubrown{tail file1}
    \item[] By default, 10 lines are displayed in \Verblubrown{head} and \Verblubrown{tail}
    \item[] To display more lines, say x, use the flag \Verblubrown{-n x} where
    \item[] Adding \Verblubrown{-f} flag to \Verblubrown{tail} will wait for additional data to be appended to the file.
%    \item[\&:] run a job in background
%    \item[CNTRL-Z:] suspend a running job
%    \item[fg:] run a suspended job in foreground
%    \item[bg:] run a suspended job in background
%    \item[CNTRL-C:] Kill a running job
%    \item[jobs:] Show list of background jobs
    \item[wait:] wait until all backgrounded jobs have completed
%    \item[kill:] kill a running job, need to provide process id
    \item[which:] shows the full path of (shell) commands
%    \item[who:] show who is logged on
%    \item[whoami:] print effective userid
%    \item[finger:] user information lookup program
    \item[whatis:] display manual page descriptions
    \item[!name:] rerun previously executed command with the same arguments as before, \Verblubrown{name <args>}.
    \item[] Note that you do not always have to type the full command \Verblubrown{name}, just the minimum unique characters (no spaces) of \Verblubrown{name} need to be entered.
    \item[] If you had entered two commands \Verblubrown{name <args>} and \Verblubrown{nbme <args>}, then to rerun \Verblubrown{name}, use the command \Verblubrown{!na{\Enter}}. 
    \item[history:] display a list of last executed commands. Optional argument \Verblubrown{m} will list the last m commands. 
    \item[] All previously executed commands will be listed with a number \Verbblue{n}.
    \item[] To rerun a command from history which has number \Verbblue{n}, run the command \Verblubrown{!n{\Enter}} 
  \end{description}
  To learn more about these commands, type \Verblubrown{man {\em command}} or  \Verblubrown{info {\em command}} on the command prompt
\end{frame}

\begin{frame}[fragile]{Filename Completion}
  \begin{itemize}
    \item Filename or Tab completion is a default feature in \Verblubrown{bash} and \Verblubrown{tcsh}.
    \item It allows to a user to automatically complete the file, directory or command name you are typing upto the next unique characters using the TAB key.
    \item Example: Your home directory contains directories \Verblubrown{Desktop}, \Verblubrown{Documents} and \Verblubrown{Downloads}.
    \item[] If you enter the command \Verblubrown{ls D{\Tab}}, you will be prompted with above the three directory names.
      \begin{lstlisting}[basicstyle=\color{lubrown}\bfseries\scriptsize\ttfamily,escapechar=\%]
[user@localhost ~]%\$% ls D%\Tab%
Desktop/ Documents/ Downloads/
[user@localhost ~]%\$% ls Do%\Tab%
Documents/ Downloads/
[user@localhost ~]%\$% ls Do
      \end{lstlisting}
  \end{itemize}
\end{frame}

\begin{frame}{Wildcards}
  \begin{itemize}
    \item *nix shells have the ability to refer to more than one file by name using special characters called Wildcards.
    \item Wildcards can be used with *nix utilities such as ls, cp, mv, rm, tar and g(un)zip.
    \item[?] match a single character
    \item[*] match zero or more characters
    \item[{[\quad]}] match list of characters in the list specified
    \item[{[!\quad]}] match characters not in the list specified
    \item Examples:
    \begin{enumerate}
      {\scriptsize
      \item \texttt{ls */*}
      \item[] list contents of all subdirectories
      \item \texttt{cp [a-z]* lower/}
      \item[] copy all files with names that begin with lowercase letters to a directory called lower
      \item \texttt{cp [!a-z]* upper\_digit/}
      \item[] copy all files with names that do not begin with lowercase letters to a directory called lower
      }
    \end{enumerate}
  \end{itemize}
\end{frame}

\begin{frame}[fragile]{How to Login to Remote Systems?}
  \begin{itemize}
    \item Most Linux/UNIX systems allow secure shell connections from other systems.
    \item[e.g.] You need to login using \Verblubrown{ssh} to the LTS HPC clusters.
    \item Usage: \Verblubrown{ssh <username>@<remote host>}
    \item Example: \Verblubrown{ssh alp514@polaris.cc.lehigh.edu}
    \item If your local machine is a UNIX-like system i.e. Linux, Mac OSX, BSD, AIX, Solaris etc and your username on the local machine is the same as that of the remote machine, then
    \item[] you can omit the \Verblubrown{<username>@} part of the argument.
    \item[] i.e. \Verblubrown{ssh <remote host>}
    \item If the remote machine is listening to ssh connections on a non default port (i.e. different from port 22) add \Verblubrown{-p <port number>} option
    \item[i.e.] \Verblubrown{ssh -p <port number> <user>@<remote host>}
    \item If you need to forward the display of an application from the remote system to your local system, add the \Verblubrown{-X} option to \Verblubrown{ssh}
    \item[] Example: \Verblubrown{ssh -X alp514@ssh.cc.lehigh.edu}
    \item Using a jump host, add \Verblubrown{-J host} option to \Verblubrown{ssh}
    \item[] Example: \Verblubrown{ssh -J alp514@ssh.cc.lehigh.edu alp514@sol.cc.lehigh.edu}
  \end{itemize}
\end{frame}

\begin{frame}[fragile,allowframebreaks]{File Transfer between two systems} 
  \begin{itemize}
    \item \Verblubrown{scp} is a command to copy files/directories between two *nix hosts over the SSH protocol.
    \item Usage: \Verblubrown{scp <options> <user>@<host>:/path/to/source/file {\textbackslash}}\\
    \Verblubrown{{\qquad}<user>@<host>:/path/to/destination/file/or/directory}
    \item[e.g.] You want to copy files between Polaris Cluster and your Linux Desktop/Laptop
    \item[] \Verblubrown{scp alp514@polaris.cc.lehigh.edu:/home/alp514/octave-tutorial.tar.gz .}
    \item[] \Verblubrown{scp -r Public alp514@polaris.cc.lehigh.edu:$\sim$/}
    \item You can omit the \Verblubrown{<user>@} part of the argument if the username is the same on both systems.
    \item You can omit the \Verblubrown{<user>@<host>:} for your local machine.
    \item Common options are \Verblubrown{-r} and \Verblubrown{-p}, same meaning as \Verblubrown{cp}.
    \item add \Verblubrown{-P <port number>} option for non default ports.
    \item Using a jump host, add \Verblubrown{-o 'ProxyJump host'} option to \Verblubrown{ssh}
    \item[] Example: \Verblubrown{scp -o 'ProxyJump alp514@ssh.cc.lehigh.edu' {\textbackslash}}\\ 
     \Verblubrown{{\qquad}alp514@sol.cc.lehigh.edu:$\sim$/myscript.log .}
      \framebreak
    \item \Verblubrown{rsync} is another utility that can be used to copy files locally and remotely.
    \item Usage: \Verblubrown{rsync <option> <source> <destination>}
    \item It is famous for its delta-transfer algorithm
    \item[i.e.] sending only the differences between  the  source  files and  the  existing  files in the destination.  
    \item Rsync is widely used for backups and mirroring and as an improved copy command for everyday use.
    \item Common options:
      {\scriptsize
      \item[] \Verblubrown{-a}: archive mode
      \item[] \Verblubrown{-r}: recurse into directories
      \item[] \Verblubrown{-v}: increase verbosity
      \item[] \Verblubrown{-z}: compress file data during the transfer
      \item[] \Verblubrown{-u}: skip files that are newer on the receiver
      \item[] \Verblubrown{-t}: preserve modification times
      \item[] \Verblubrown{-n}: dry-run, perform a trial run with no changes made
      }
      \item Example: \Verblubrownp{rsync -avtzu corona.cc.lehigh.edu:$\sim$/* .}
      \item If you are a user on National Supercomputing resource such as XSEDE, NERSC, OSG, etc, there are other transfer tools such as globus toolkit (gridftp) and bbcp which provide higher bandwidth and parallel file transfers.
  \end{itemize}
\end{frame}

\begin{frame}[fragile,allowframebreaks]{Compressing and Archiving Files}
  \begin{itemize}
    \item Quite often you need to compress and uncompress files to reduce storage usage or bandwidth while transferring files.
    \item *nix systems have built-in utilities to compress/uncompress files
      \begin{columns}
        \column{0.4\textwidth}
        \begin{exampleblock}{Compress}
          \Verblubrown{gzip, zip, bzip2}\\
          \Verblubrown{gzip README{\Enter}}
        \end{exampleblock}
        \column{0.4\textwidth}
        \begin{exampleblock}{Uncompress}
          \Verblubrown{gunzip, unzip, bunzip2}\\
          \Verblubrown{gunzip README.gz{\Enter}}
        \end{exampleblock}
      \end{columns}
    \item Gzipped files have an extension \Verblubrown{.gz,.z} or \Verblubrown{.Z}
    \item zipped files have an extension \Verblubrown{.Zip} or \Verblubrown{.zip}
    \item Bzipped files have an extension \Verblubrown{.bz2, .bz}
    \item To compress/uncompress files recursively, use the \Verblubrown{-r} option.
    \item To overwrite files while compressing/uncompressing, use the \Verblubrown{-f} option.
      \framebreak
    \item *nix provides the \Verblubrown{tar} package to create and manipulate streaming archive of files.
    \item Usage: \Verblubrown{tar <options> <file> <patterns>}
    \item[] \Verblubrown{file} is the name of the tar archive file, usually with extension \Verblubrown{.tar}
    \item[] \Verblubrown{patterns} are pathnames for files/directories being archived
    \item Common options
    \item[] \Verblubrown{-c}: create an archive file
    \item[] \Verblubrown{-x}: extract to disk from archive
    \item[] \Verblubrown{-t}: list contents of archive
    \item[] \Verblubrown{-z}: filter the archive through gzip (adds/requires extension .gz)
    \item[] \Verblubrown{-j}: filter the archive through bzip2 (adds/requires extension .bz2)
    \item[] \Verblubrown{-v}: verbosely list files processed
    \item[] \Verblubrown{-f}: read the archive from or write the archive to the specified file
    \item[e.g.] \Verblubrown{tar -cvzf myhome.tar.gz \$\{HOME\}/*}
    \item This becomes useful for creating a backup of your files and directories that you can store at some storage facility e.g. external disk
  \end{itemize}
\end{frame}

\section{Editors}
\begin{frame}
  \frametitle{File Editing}
  \begin{itemize}
    \item The two most commonly used editors on Linux/Unix systems are:
    \begin{enumerate}
      {\scriptsize
      \item \Verblubrown{vi} or \Verblubrown{vim} (vi improved)
      \item \Verblubrown{emacs}
      }
    \end{enumerate}
    \item \Verblubrown{vi/vim} is installed by default on Linux/Unix systems and has only a command line interface (CLI).
    \item \Verblubrown{emacs} has both a CLI and a graphical user interface (GUI).
    \item[$\vardiamond$] If \Verblubrown{emacs} GUI is installed then use \Verblubrown{emacs -nw} to open file in console.
    \item Other editors that you may come across on *nix systems
    \item[] \Verblubrown{kate}: {default editor for KDE.}
    \item[] \Verblubrown{gedit}: {default text editor for GNOME desktop environment.}
    \item[] \Verblubrown{gvim}: {GUI version of }\Verblubrown{vim}
    \item[] \Verblubrown{pico}: {console based plain text editor }
    \item[] \Verblubrown{nano}: {GNU.org clone of }\Verblubrown{pico}
    \item[] \Verblubrown{kwrite}: {editor by KDE.}
  \end{itemize}
\end{frame}

\begin{frame}[allowframebreaks,fragile]
  \frametitle{Editor Cheatsheets}
% Page 1
  \begin{itemize}
    \item \textbf{\color{lubrown}vi/vim} and \textbf{\color{lubrown}emacs} are the two most popular *nix file editors.
    \item Which one to use is up to you.
    \item \textbf{\color{lubrown}vi/vim} has two modes:
    \begin{enumerate}
      {\scriptsize
        \item Editing mode
        \item Command mode
      }
    \end{enumerate}
    \item \textbf{\color{lubrown}emacs} has only one mode as in any editor that you use.
  \end{itemize}
  \begin{center}
    \taburulecolor{lublue}
    \begin{tabular}{a|b}
      %\rowcolor{lublue}\multicolumn{2}{c}{Insert/Appending Text}\\
      \rowcolor{lublue}{Insert/Appending Text} & {vi}\\
       \rowcolor{lulime}insert at cursor & \texttt{i}\\
       \rowcolor{lulime!50}insert at beginning of line& \texttt{I}\\
       \rowcolor{lulime}append after cursor& \texttt{a}\\
       \rowcolor{lulime!50}append at end of line& \texttt{A}\\
       \rowcolor{lulime}newline after cursor in insert mode& \texttt{o}\\
       \rowcolor{lulime!50}newline before cursor in insert mode& \texttt{O}\\
       \rowcolor{lulime}append at end of line& \texttt{ea}\\
       \rowcolor{lulime!50}exit insert mode& \texttt{ESC}\\
   \end{tabular}
  \end{center}

% Page 2

  \begin{center}
    \taburulecolor{lublue}
    \begin{tabular}{a|b|b}
      \rowcolor{lublue}Cursor Movement & vi & emacs \\
      \rowcolor{lulime}move left       & h  & \Ctrl-b \\
      \rowcolor{lulime!50}move down       & j  & \Ctrl-n \\
      \rowcolor{lulime}move up         & k  & \Ctrl-p \\
      \rowcolor{lulime!50}move right      & l  & \Ctrl-f \\
      \rowcolor{lulime}jump to begining of line & \Verb|^| & \Ctrl-a \\
      \rowcolor{lulime!50}jump to end of line      & \$       & \Ctrl-e \\
      \rowcolor{lulime}goto line \texttt{n}     & nG       & \Esc x goto-line \Enter\,n \\
      \rowcolor{lulime!50}goto top of file         & 1G       & \Esc $<$ \\
      \rowcolor{lulime}goto end of file         &  G       & \Esc $>$ \\
      \rowcolor{lulime!50}move one page up         & \Ctrl-u  & \Esc v \\
      \rowcolor{lulime}move one page down       & \Ctrl-d  & \Ctrl-v \\
    \end{tabular}
  \end{center}

% Page 3

  \begin{center}
    \taburulecolor{lublue}
    \begin{tabular}{a|b|b}
      \rowcolor{lublue}File Manipulation & vi & emacs \\
      \rowcolor{lulime}edit/open file \textit{file}  & :e \textit{file} & \Ctrl-x \Ctrl-f \textit{file}\\
      \rowcolor{lulime!50}insert file \textit{file}  & :r \textit{file} & \Ctrl-x i \textit{file}\\
      \rowcolor{lulime}save file            & :w       & \Ctrl-x\Ctrl-s \\
      \rowcolor{lulime!50}save file and exit   & :wq, ZZ (if file exists) & \\
      \rowcolor{lulime}quit                 & :q       & \Ctrl-x\Ctrl-c \\
      \rowcolor{lulime!50}quit without saving  & :q!      & \\
    \end{tabular}
  \end{center}

  \begin{center}
    \taburulecolor{lublue}
    \begin{tabular}{a|b|b}
      \rowcolor{lublue}Text Manipulation & vi & emacs \\
      \rowcolor{lulime}delete a line        & dd      & \Ctrl-a\Ctrl-k \\
      \rowcolor{lulime!50}delete \texttt{n} lines & \texttt{n}dd  & \Ctrl-a\Esc n\Ctrl-k \\
      \rowcolor{lulime}paste deleted line after cursor & p & \Ctrl-y \\
      \rowcolor{lulime!50}paste deleted line before curosr & P & \\
      \rowcolor{lulime}undo edit            & u       & \Ctrl-\_ or  \Ctrl-x u \\
      \rowcolor{lulime!50}delete cursor to end of line & D & \Ctrl-k \\
      \rowcolor{lulime}search forward for \textit{patt} & \texttt{$\backslash$\textit{patt}} & \Ctrl-s \textit{patt} \\
      \rowcolor{lulime!50}search backward for \textit{patt} & ?\textit{patt} & \Ctrl-r \textit{patt} \\
      \rowcolor{lulime}search again forward (backward) & n & \Ctrl-s(r) \\
      \rowcolor{lulime!50}replace a character  & r &\\
      \rowcolor{lulime}join next line to current  & J &  \\
      \rowcolor{lulime!50}change a line  & cc &\\
      \rowcolor{lulime}change a word  & cw & \\
      \rowcolor{lulime!50}change to end of line  & c\$ & \\
      \rowcolor{lulime}delete a character  & x & \Ctrl-d\\
      \rowcolor{lulime!50}delete a word  & dw & \Esc-d \\
    \end{tabular}
  \end{center}

  \begin{center}
    \taburulecolor{lublue}
    \begin{tabular}{a|b|b}
      \rowcolor{lublue}Window Management & vi & emacs \\
      \rowcolor{lulime}split window horizontally  & :split or \Ctrl-w s & \Ctrl-x 2\\
      \rowcolor{lulime!50}split window vertically & :vsplit or \Ctrl-w v & \Ctrl-x 3\\
      \rowcolor{lulime}switch windows & \Ctrl-w w & \Ctrl-x o\\
    \end{tabular}
  \end{center}

  \begin{itemize}
    \item Do a google search for more detailed cheatsheets
    \item[\texttt{vi}] \url{https://www.google.com/search?q=vi+cheatsheet}
    \item[\texttt{emacs}] \url{https://www.google.com/search?q=emacs+cheatsheet}
  \end{itemize}
  \begin{block}{More on the \textbf{set -o} command}
    \begin{itemize}
      \item The \textbf{\color{lubrown}set -o} command can be used to change the command line editor mode among other things (Do \textbf{\color{lubrown}man set\Enter} to find out more)
      \begin{enumerate}
        {\scriptsize
        \item \textbf{\color{lubrown}set -o emacs}: emacs style in-line editor for command entry, this is the default
        \item \textbf{\color{lubrown}set -o vi}: vi style in-line editor for command entry.
        }
      \end{enumerate}
    \end{itemize}
  \end{block}
\end{frame}

\section{File \& Job Manipulation}
\begin{frame}
  \frametitle{I/O Redirection}
  \begin{itemize}
    \item There are three file descriptors for I/O streams
    \begin{enumerate}
      {%\scriptsize
        \item[STDIN]: Standard Input
        \item[STDOUT]: Standard Output
        \item[STDERR]: Standard Error
      }
    \end{enumerate}
    \item 1 represents STDOUT and 2 represents STDERR
    \item I/O redirection allows users to connect applications
    \begin{itemize}
      {%\scriptsize
      \item[$<$]: connects a file to STDIN of an application
      \item[$>$]: connects STDOUT of an application to a file
      \item[$> >$]: connects STDOUT of an application by appending to a file
      \item[$|$]: connects the STDOUT of an application to STDIN of another application.
      }
    \end{itemize}
    \item Examples:
    \begin{enumerate}
      {%\scriptsize
      \item write STDOUT to file: \Verblubrown{ls -l > ls-l.out }
      \item write STDERR to file: \Verblubrown{ls -l 2> ls-l.err }
      \item write STDOUT to STDERR: \Verblubrown{ls -l 1>\&2 }
      \item write STDERR to STDOUT: \Verblubrown{ls -l 2>\&1 }
      \item send STDOUT as STDIN: \Verblubrownp{ls -l | wc -l}
      }
    \end{enumerate}
  \end{itemize}
\end{frame}

\begin{frame}[fragile, allowframebreaks]
  \frametitle{File Permissions}
  \begin{itemize}
    \item Since *NIX OS's are designed for multi user environment, it is necessary to restrict access of files to other users on the system.
    \item In *NIX OS's, you have three types of file permissions
    \begin{enumerate}
      {%\scriptsize
        \item read (r)
        \item write (w)
        \item execute (x)
      }
    \end{enumerate}
    \item for three types of users
    \begin{enumerate}
      {%\scriptsize
        \item user (u)
        \item group (g)
        \item world (o) i.e. everyone else who has access to the system
      }
    \end{enumerate}
    \begin{Verbatim}[formatcom=\color{lubrown},fontsize=\scriptsize]
[user@localhost ~]$ ls -l
total 44
drwxr-xr-x. 2 user user 4096 Jan 28  2013 Desktop
drwxr-xr-x. 2 user user 4096 Jan 28  2013 Documents
drwxr-xr-x. 2 user user 4096 Jan 28  2013 Downloads
-rwxr-xr-x. 1 user user   32 Sep 11 11:57 hello
drwxr-xr-x. 2 user user 4096 Jan 28  2013 Music
drwxr-xr-x. 2 user user 4096 Jan 28  2013 Pictures
drwxr-xr-x. 2 user user 4096 Jan 28  2013 Public
-rw-rw-r--. 1 user user 3047 Sep 11 11:48 README
drwxr-xr-x. 1 root root 4216 Jan 22 16:17 Shared
drwxr-xr-x. 2 user user 4096 Jan 28  2013 Templates
lrwxrwxrwx. 1 user user    5 Jan 23 08:17 test -> hello
drwxr-xr-x. 2 user user 4096 Jan 28  2013 Videos
[user@localhost ~]$ 
    \end{Verbatim}
    \item The first character signifies the type of the file
    \item[] \Verblubrown{d} for directory
    \item[] \Verblubrown{l} for symbolic link
    \item[] \Verblubrown{-} for normal file
    \item The next three characters of first triad signifies what the owner can do
    \item The second triad signifies what group member can do
    \item The third triad signifies what everyone else can do
      \begin{gather*}
        d\underbrace{rwx}_{u}\overbrace{r-x}^g\underbrace{r-x}_o
      \end{gather*}
    \item Read carries a weight of 4
    \item Write carries a weight of 2
    \item Execute carries a weight of 1
    \item The weights are added to give a value of 7 (rwx), 6(rw), 5(rx) or 3(wx) permissions. 
    \item \Verblubrown{chmod} is a *NIX command to change permissions on a file
    \item[] Usage: \Verblubrown{chmod <option> <permissions> <file or directory name>}
    \item To give user rwx, group rx and world x permission, the command is
    \item[] \Verblubrown{chmod 751 filename}
    \framebreak
    \item Instead of using numerical permissions you can also use symbolic mode
  \end{itemize}
  \begin{description}
    \item[u/g/o or a] user/group/world or all i.e. ugo
    \item[+/-] Add/remove permission
    \item[r/w/x] read/write/execute
  \end{description}
  \begin{itemize}
    \item Give everyone execute permission: 
    \item[] \Verblubrown{chmod a+x hello.sh }
    \item[] \Verblubrown{chmod ugo+x hello.sh}
    \item Remove group and world read \& write permission: 
    \item[] \Verblubrown{chmod go-rw hello.sh}
    \item To change permissions recursively in a directory, use the option \Verblubrown{-R} (can also be used in the following two commands)
    \item[] \Verblubrown{chmod -R 755 \$\{HOME\}/*}
    \item[] What is the permission on \$\{HOME\}?
    \framebreak
    \item The \Verblubrown{chgrp} command is used to change the group ownership between two groups that you are a member of.
    \item[] Usage: \Verblubrown{chgrp <option> <new group> <file or directory name>}
    %\item[e.g.] Suppose your default group on LSU HPC is \Verbblue{users} and your advisor requested sysadmins to create a group \Verbblue{abc} for collaborative research among say 10 researchers. 
    \item You can use the \Verblubrown{chgrp} command to change the ownership of your files from the \Verbblue{users} group to \Verbblue{abc} group.
    \item[] Example: \Verblubrown{chgrp -R abc collaborative-work-dir}
    \item The \Verblubrown{chown} command is used to  change the owner of a file.
    \item \Verblubrown{chown} can only be executed by the superuser, to prevent users simply changing ownership of files that aren't theirs to access. 
    \item[] Usage: \Verblubrown{chown <new owner>[:<group name>] <file or directory name>}
  \end{itemize}
\end{frame}

\begin{frame}[fragile,allowframebreaks]{Processes and Jobs}
  \begin{itemize}
    \item A process is an executing program identified by a unique PID
    \item[$\mybigstar$] To see information about your running processes and their PID and status,
    \item[] \Verblubrown{ps{\Enter}}
    \item A process may be in foreground, background or be suspended.
    \item Processes running in foreground, the command prompt is not returned until the current process has finished executing.
    \item If a job takes a long time to run, put the job in background in order to obtain the command prompt back to do some other useful work
    \item There are two ways to send a job into the background:
    \begin{enumerate}
      {%\scriptsize
        \item Add an ampersand \Verblubrown{\&} to the end of your command to send it into background directly.
        \item[] \Verbblue{firefox \&{\Enter}}
        \item First suspend the job using \Verbblue{{\Ctrl}Z} and then type \Verbblue{bg} at the command prompt.
        \item If you type \Verbblue{fg} then the job will run in foreground and you will lose the command prompt.
      }
    \end{enumerate}
    \framebreak
    \item When a process is running, background or suspended, it will be entered onto a list along with a job number (not PID)
    \item[] \Verblubrown{jobs{\Enter}}
    \item To restart a suspended job in foreground or background, type
    \item[] \Verblubrown{fg \%jobnumber} where \Verbblue{jobnumber} is a number greater than 1, or,
    \item[] \Verblubrown{bg \%jobnumber} 
    \item To kill or terminate a process:
    \begin{enumerate}
      {%\scriptsize
        \item Job running in foreground: enter \Verbblue{{\Ctrl}C}
        \item Job whose PID you know
        \item[] \Verblubrown{kill PID{\Enter}}
        \item Job whose \Verbblue{jobnumber} you know (from \Verblubrown{jobs} command)
        \item[] \Verblubrown{kill \%jobnumber{\Enter}}
      }
    \end{enumerate}
    \item The \Verblubrown{kill} command can take options specific to UNIX signals
    \item The most common option is \Verblubrown{-9} for the \Verblubrown{SIGKILL} signal
    \item \Verblubrown{pstree}: display a tree of processes
    \item \Verblubrown{pkill}: kill process by its name, user name, group name, terminal, UID, EUID, and GID.
  \end{itemize}
\end{frame}

\section{Environment \& Environment Variables}
\begin{frame}
  \frametitle{Start Up Scripts}
  \begin{itemize}
    \item When you login to a *NIX computer, shell scripts are automatically loaded depending on your default \textbf{\color{lubrown}shell}
    \item \textbf{\color{lubrown}sh,ksh}
    \begin{enumerate}
        \item \texttt{\color{blue}/etc/profile}
        \item \texttt{\color{blue}\$HOME/.profile}
    \end{enumerate}
    \item \textbf{\color{lubrown}bash}
    \begin{enumerate}
        \item \texttt{\color{blue}/etc/profile}, login terminal only
        \item \texttt{\color{blue}/etc/bashrc} or \texttt{\color{blue}/etc/bash/bashrc}
        \item \texttt{\color{blue}\$HOME/.bash\_profile}, login terminal only
        \item \texttt{\color{blue}\$HOME/.bashrc}
    \end{enumerate}
    \item \textbf{\color{lubrown}csh,tcsh}
    \begin{enumerate}
        \item \texttt{\color{blue}/etc/csh.cshrc}
        \item \texttt{\color{blue}\$HOME/.tcshrc}
        \item \texttt{\color{blue}\$HOME/.cshrc} if .tcshrc is not present
    \end{enumerate}
    \item The \texttt{\color{blue}.bashrc, .tcshrc, .cshrc, .bash\_profile} are script files where users can define their own aliases, environment variables, modify paths etc.
    \item e.g. the \textbf{\color{lubrown}alias} command covered earlier can be put in one of these script files depending on your \textbf{\color{lubrown}shell}
  \end{itemize}
\end{frame}

\begin{frame}[fragile, allowframebreaks]
  \frametitle{Examples}
  \begin{lstlisting}[language=bash,basicstyle=\tiny\ttfamily]
# .bashrc

# Source global definitions
if [ -f /etc/bashrc ]; then
        . /etc/bashrc
fi

# User specific aliases and functions
alias c="clear"
alias rm="/bin/rm -i"
alias psu="ps -u apacheco"
alias em="emacs -nw"
alias ll="ls -lF"
alias la="ls -al"
export PATH=/home/apacheco/bin:${PATH}
export g09root=/home/apacheco/Software/Gaussian09
export GAUSS_SCRDIR=/home/apacheco/Software/scratch
source $g09root/g09/bsd/g09.profile

export TEXINPUTS=.:/usr/share/texmf//:/home/apacheco/LaTeX//:${TEXINPUTS}
export BIBINPUTS=.:/home/apacheco/TeX//:${BIBINPUTS}
  \end{lstlisting}

  \begin{lstlisting}[language=csh,basicstyle=\tiny\ttfamily]
# .tcshrc

# User specific aliases and functions
alias c clear
alias rm "/bin/rm -i"
alias psu "ps -u apacheco"
alias em "emacs -nw"
alias ll "ls -lF"
alias la "ls -al"
setenv PATH "/home/apacheco/bin:${PATH}"
setenv g09root "/home/apacheco/Software/Gaussian09"
setenv GAUSS_SCRDIR "/home/apacheco/Software/scratch"
source $g09root/g09/bsd/g09.login

setenv TEXINPUTS ".:/usr/share/texmf//:/home/apacheco/LaTeX//:${TEXINPUTS}"
setenv BIBINPUTS ".:/home/apacheco/TeX//:${BIBINPUTS}"
  \end{lstlisting}
\end{frame}

\begin{frame}[fragile,allowframebreaks]
  \frametitle{Variables}
  \begin{itemize}
    \item *nix also permits the use of variables, similar to any programming language such as C, C++, Fortran etc
    \item A variable is a named object that contains data used by one or more applications. 
    \item There are two types of variables, Environment and User Defined and can contain  a number, character or a string of characters.
    \item Environment Variables provides a simple way to share configuration settings between multiple applications and processes in Linux.
    \item By Convention, enviromental variables are often named using all uppercase letters
    \item[e.g.] \Verblubrown{PATH, LD\_LIBRARY\_PATH, LD\_INCLUDE\_PATH, TEXINPUTS,} etc
    \item To reference a variable (environment or user defined) prepend \$ to the name of the variable
    \item[e.g.] \Verblubrown{\$PATH, \$LD\_LIBRARY\_PATH}
    \framebreak
  \end{itemize}
%\end{frame}
%
%\begin{frame}[allowframebreaks]
%  \frametitle{Environment Variable}
  \begin{itemize}
    \item The command \Verblubrown{printenv} list the current environmental variables.
    \item[$\mybigstar$] {Type \Verblubrown{printenv} on your command prompt to list all environment variables in your current session.}
    \item The command \Verblubrown{env} is used to either print a list of environment variables or run another utility in an altered environment without having to modify the currently existing environment.
    \item[$\mybigstar$] {Type \Verblubrown{env SHELL=/bin/tcsh xterm} to start an xterm session in tcsh}
    \item[$\vardiamond$] {To execute the above command successfully, you need to be in GUI mode on the virtual OS or logged into a remote systems with X-Forwarding enabled.}
  \end{itemize}
  \framebreak
  \begin{description}
    \item[PATH:] A list of directory paths.
    \item[HOME:] indicate where a user's home directory is located in the file system.
    \item[PWD:] contains path to current working directory.
    \item[OLDPWD:] contains path to previous working directory.
    \item[TERM:] specifies the type of computer terminal or terminal emulator being used
    \item[SHELL:] contains name of the running, interactive shell.
    \item[PS1:] default command prompt
    \item[PS2:] secondary command prompt
    \item[LD\_LIBRARY\_PATH:] colon-separated set of directories where libraries should be searched for first
    \item[HOSTNAME:] The systems host name
    \item[USER:] Current logged in user's name
    \item[DISPLAY:] Network name of the X11 display to connect to, if available.
%    \item[LD\_INCLUDE\_PATH:] 
  \end{description}
  \framebreak
  \begin{itemize}
    \item You can edit the environment variables.
    \item Command to do this depends on the shell
    \item[$\mybigstar$] To add your bin directory to the \Verblubrown{PATH} variable
    \item[] \Verblubrown{sh/ksh/bash}: \Verblubrown{export PATH=\$\{HOME\}/bin:\$\{PATH\}}
    \item[] \Verblubrown{csh/tcsh}: \Verblubrown{setenv PATH \$\{HOME\}/bin:\$\{PATH\}}
    \item[$\mybigstar$] Note the syntax for the above commands
    \item[$\mybigstar$] \Verblubrown{sh/ksh/bash:} {no spaces except between \Verblubrown{export} and \Verblubrown{PATH}}
    \item[$\mybigstar$] \Verblubrown{csh,tcsh:} {no = sign, just a space between \Verblubrown{PATH} and the absolute path}
    \item[$\mybigstar$] \Verblubrown{all shells:} {colon(:) to separate different paths and the variable that is appended to}
    \item \textbf{Yes, the order matters.} If you have a customized version of a software say perl in your home directory, if you append the perl path to \Verblubrown{PATH} at the end, your program will use the system wide perl not your locally installed version.
      \framebreak
    \item Rules for Variable Names
    \begin{enumerate}
      {%\scriptsize
        \item Variable names must start with a letter or underscore
        \item Number can be used anywhere else
        \item DO NOT USE special characters such as \texttt{@, \#, \%, \$}
        \item Case sensitive
        \item Examples
        \begin{itemize}
          {%\scriptsize
          \item Allowed: VARIABLE, VAR1234able, var\_name, \_VAR
          \item Not Allowed: 1VARIABLE, \%NAME, \$myvar, VAR@NAME 
          }
        \end{itemize}
      }
    \end{enumerate}
    \item Assigning value to a variable
    \begin{center}
      \begin{tikzpicture}
        \node (tbl) {
          \begin{tabularx}{0.65\textwidth}{ccc}
            \arrayrulecolor{black}
            \textcolor{white}{\textbf{Type} }& \textcolor{white}{\textbf{sh,ksh,bash}} &\textcolor{white}{\textbf {csh,tcsh}}\\
            Shell \rule{0pt}{3.5ex} & name=value & set name = value \\
            Environment & export name=value & setenv name value \\
            [1.0ex]
        \end{tabularx}};
        \begin{pgfonlayer}{background}
          \draw[rounded corners,top color=lupurple,bottom color=lupurple,
            draw=lubrown!30] ($(tbl.north west)+(0.14,0)$)
          rectangle ($(tbl.north east)-(0.13,0.9)$);
          \draw[rounded corners,top color=lulime,bottom color=lulime,draw=lulime]
          ($(tbl.north east)-(0.13,0.6)$)
          rectangle ($(tbl.south west)+(0.13,0.2)$);
        \end{pgfonlayer}
      \end{tikzpicture}
    \end{center}
    \item \Verblubrown{sh,ksh,bash} THERE IS NO SPACE ON EITHER SIDE OF =
    \item \Verblubrown{csh,tcsh} space on either side of = is allowed for the \Verblubrown{set} command
    \item \Verblubrown{csh,tcsh} There is no = in the \Verblubrown{setenv} command
  \end{itemize}
%  \framebreak
%  \begin{exampleblock}{Exercise}
%    \begin{itemize}
%      \item Create two shell variables containing
%      \begin{enumerate}
%        {%\footnotesize
%          \item your name
%          \item[e.g.] MYNAME=Alex
%          \item a standard greeting
%          \item[e.g.] Greet=Hello
%        }
%      \end{enumerate}
%      \item We'll make use of this variables in a few slides when we learn some basic commands.
%    \end{itemize}
%  \end{exampleblock}
\end{frame}

\begin{frame}
  \frametitle{}
  \color{blue}{
  \begin{center}
    {\fontsize{40}{60}\selectfont The End}\\
    \vspace{1cm}
    {\fontsize{20}{30}\selectfont Any Questions?}\\
    \vspace{0.5cm}
    {\fontsize{15}{30}\selectfont
        Subscribe to \href{http://groups.google.com/a/lehigh.edu/group/lehigh-linux-list/}{Lehigh Linux Google Group} via email to\\
        \url{lehigh-linux-list+subscribe@lehigh.edu}\\
    }
%    \vspace{0.5cm}
%    {\fontsize{15}{30}\selectfont
%      \color{red!80!black}{Sep 22: Research Computing Resources at Lehigh}\\
%      \color{red!80!black}{Sep 29 \& Oct 6: Shell Scripting}\\
%    }
%    \vspace{0.5cm}
%    {\fontsize{15}{20}\selectfont Survey: \color{red!90!white}{\url{http://www.hpc.lsu.edu/survey}}}
  \end{center}
  }
\end{frame}

\end{document}

